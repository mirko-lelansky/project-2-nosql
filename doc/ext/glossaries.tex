\newacronym{AJAX}{AJAX}{Asynchronous JavaScript And XML}
\newacronym{AOF}{AOF}{Append Only Files}
\newacronym{API}{API}{Application Programming Interface}
\newacronym{JSON}{JSON}{JavaScript Object Notation}
\newacronym{ORM}{ORM}{Objekt-Relationalen-Mapper}
\newacronym{RAM}{RAM}{Random-Access-Memory}
\newacronym{RDB}{RDB}{Redis Database File}
\newacronym{XML}{XML}{extensible Markup Language}
\newglossaryentry{ACID}{name={Atomicity, Consistency, Isolation and Durability},
description={ACID steht für Atomarität, Konsistenz, Isolation und
Dauerhaftigkeit. Atomarität meint in diesem Zusammenhang, das eine Liste
von Operationen entweder vollständig oder gar nicht durchgeführt wird. Konsistenz
meint die Eigenschaft, dass die Datenbank sich nach einer Liste von Operationen
wieder in einem konsistenten Zustand befindet, wenn sie voher sich in einem
konsistenten Zustand befand. Isolation meint die Fähigkeit, das es keine
unkontrollierten Operationen auf den Daten gibt durch zwei parallele Prozesse.
Die Dauerhaftigkeit sagt aus, dass geschriebene Daten auch nach einem Ausfall
des Systems noch dauerhaft vorhanden sind.}}
\newglossaryentry{BASE}{name={Basically Available, Soft State,
Eventually Consistent}, description={BASE ist der Gegenpart zu \gls{ACID} für
NoSql-Systeme. Bei BASE geht es darum die Konsistenz der Verfügbarkeit
unterzuordnen, da es in verteilten Datenbanken nicht möglicht ist mit ACID die
geforderte Verfügbarkeit zu gewährleisten. Ein System, das nach BASE arbeitet
muss dabei immer die Verfügbarkeit vom \gls{CAP}-Theorem sicherstellen.
Soft States bedeutet, dass sich verschiedene Versionen eines Datenbestandes auf
dem einzelnen Knoten eines Clusters befinden können und nach einer gewissen Zeit
auf allen Knoten die letzte Version vorhanden ist und das System sich dann
wieder im Hard State befindet. Eventually Consistent meint, dass das System
nach einer gewissen Zeit sich wieder in einem konsistenten Zustand befindet.}}
\newglossaryentry{BigData}{name={Massendaten}, description={Der Begriff BigData
bezeichnet eine Datenmenge, die mit herkormlichen Mitteln nicht verarbeitet
werden kann, weil sie zu groß, zu unstrukturiert, zu komplex sind oder in
Echtzeit verarbeitet werden müssen. Dies hängt mit den \gls{5V's} zusammen.}}
\newglossaryentry{CAP}{name={Consistency, Availability, Partition Tolerance},
description={CAP steht für Konsistenz, Verfügbarkeit und Ausfalltoleranz und ist
ein Theorem für verteielte Systeme von Eric Brewer. Das Theorem sagt aus, dass
man lediglich zwei der drei Eigenschaften gleichzeitig einhalten kann, die
dritte wird man nie einhalten können. Verfügbarkeit ist dabei die Antwortzeit
eines Systems auf eine Anfrage. Konsistenz meint hierbei, dass nach einer
Änderung der Daten auf einem Knoten, diese an die anderen Knoten weitergereicht
wird. Ausfalltoleranz meint, dass das System auch bei Verlust von Nachrichten
noch weiterarbeit. \cite{Brewer2000}}}
\newglossaryentry{5V's}{name={5V's}, description={Die 5V's stehen für Volume,
Velocity, Variaty, Value und Validaty. Dabei steht Volume für die Datenmenge,
Velocity für Zeit mit der die Daten verarbeitet werden müssen, Variaty steht
für die Struktur der Daten, Value steht für den zu erzeugenden Mehrwert und
Validaty steht für die Datenqualität.}}
