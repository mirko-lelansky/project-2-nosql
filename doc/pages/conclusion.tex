\chapter{Zusammenfassung}
In diesem Kapitel geht es darum, die Ergebnisse nochmals zusammenzufassen und
einen Ausblick auf darauf aufbauende Möglichkeiten zu liefern.

\section{Fazit}
Die drei NoSql-Systeme wurden hinsichtlich ihrer Eigenschaften untersucht.
Dabei wurden die jeweiligen System kurz vorgestellt und ihre wichtigsten Punkte
erläutert. Danach wurden die ausgesuchten Eigenschaften verglichen. Dabei kam
heraus, dass Memcached und Redis sehr ähnlich konzipierte System sind. Dies ist
auch klar, da beide oft als Caches benutzt werden und Voldemort ist mehr wie
ein verteielter Speicher. Da alle Systeme die zu untersuchten Eigenschaften
unterschiedlich umgesetzt haben, ist es nicht möglich einen Favoriten zu
bestimmen. Welches System am besten geeignet ist, hängt vom geplanten
Einsatzzweck ab. Für Memcached wurde ein neues Performance-Tool entwickelt,
auch wenn nicht alle geplannten Funktionen umgesetzt wurden.

Der geplannte Anwendungsfall wurde trotzt der Schwierigkeiten von Voldemort
umgesetzt und je nach Bedarf können alle drei NoSql-Systeme benutzt werden.
Jedoch haben Memcached und Redis gegenüber Voldemort den Vorteil, dass sie schon
eine längere Entwicklungszeit hinter sich haben und deshalb sowohl die
NoSql-System, wie auch Bibliotheken und Tools stabiler sind. Deshablb würde
ich nur Memcached und Redis im produktiven Einsatz empfehlen, da die Bibliothek
bei beiden Systemen ausgereifter sind, was einen Einsatz in unterschiedlichen
Systemen ermöglicht. Voldemort ist hier noch zu sehr auf die Java-Welt fixiert
und ist auch nur eingeschrängt verfügbar. Während Memcached, Redis und die
dazugehörigen Bibliotheken bei Linux über die Paketmanager verfügbar sind, bzw.
je nach Art der Programmiersprache auf den jeweiligen Seiten veröffentlicht sind,
ist dies bei Voldemort zum Zeitpunkt der Untersuchung nicht der Fall gewesen.
Es gibt jedoch schon Plannungen in die Richtung. Dies würde auch eine größere
Entwicklergemeinde erzeugen und den Unternehmen, die Möglichkeit geben über
mögliche Einsatzszenarien nachzudenken.

\section{Ausblick}
Für die zukünftigen Schritte ergeben sich zwei Möglichkeiten. Entweder man
entwickelt, das Benchmark-Tool für Memcached weiter oder man entwickelt ein
ähnliches Tool für Voldemort um dann zum Beispiel den gemeinsamen Nenner aller
drei Systeme zu vergleichen nämlich einen Text als Schlüssel und Wert zu
verwenden, welchen man in der Länge variable hällt. Dabei müsst das Tool für
Voldemort jedoch derzeit noch entweder in Java entwickelt werden oder über
entsprechende Bindings, wie \gls{JNI}, Jython, JRuby,~\dots{} verfügen.

Die andere Möglichkeit besteht, darin die Webseite weiter zu entwickeln und
dann darauf gewisse Lasttests zu fahren. Dadurch könnte man feststellen, welche
System sich im Echtzeitbetrieb besser verhalten. Dazu müsste man jedoch dann
konsequent die Low-Level-Cache-API von Django benutzen, damit man auch die
Funktionalität von Redis benutzen kann um Listen und Mengen zu speichern.
Außerdem konnte man die in der Implementierung angesprochenen Probleme noch
umsetzen, wie die nichtvorhandene Benutzerregistrierung.
