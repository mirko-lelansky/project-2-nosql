\chapter{Einleitung}

In der heutigen Welt werden täglich große Datenmengen erzeugt. Sei es im
privaten Umwelt durch die Benutzung von Social-Media-Plattformen wie Facebook,
Twitter, \ldots{} oder in der Wirtschaft durch Börsendaten, medizinische
Bilddaten, \ldots{} . Im Zuge der zunehmenden Vernetzung von
Alltagsgegenständen wie Fernseher, Kühlschrank, \ldots{}, welches man auch als
IoT\footnote{Internet of the Thinks} bezeichnet, stieg die täglich erzeugte
Datenmenge nocheinmal sehr stark an.

Diese Daten können dabei strukturiert, semi-strukturiert oder unstrukturiert
sein. Strukturierte Daten können sehr leicht von Maschinen verarbeitet werden,
da sie über eine fest vorgegeben Struktur verfügen. Im Gegensatz dazu haben
semi-strukturiete Daten eine lose Struktur, dies bedeuted, dass definiert ist,
wie einzelne Bausteine auszusehen haben, jedoch nicht wie das Dokument aus den
definierten Bausteinen entsteht. Über unstrukturierte Daten kann lediglich die
Aussage treffen, um welchen Typ von Daten es sich handelt, also zum Beispiel,
ob es ein Bild ist.

Diese riesigen unterschiedliche strukturierten Daten müssen effizient
gespeichert und verarbeitet werden, damit ein Mehrwert entstehen kann.
Relationale Datenbanksysteme stoßen dabei schnell an ihre Grenzen. Für solche
Zwecke ist es besser alternative Systeme einzusetzen wie NoSql-Systeme.

\section{Motivation}
\section{Ziel der Arbeit}
\section{Aufbau der Arbeit}
