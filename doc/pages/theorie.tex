\chapter{Theoretische Grundlagen}
In diesem Kapitel werden die für die Realisierung benötigten Technologien und
Methoden beschrieben.

Dazu wird zuerst erklärt, was man unter NoSql versteht und eine kurze Zeitlinie
der Entwicklung. Anschließend werden die verschiedenen Arten von NoSql und
speziell von Schlüssel-Wert-Systemen beschrieben.

Danach folgt der Vergleich von Redis, Memcached und Voldemort. Dazu werden
zuerst die jeweiligen Systeme erläutert. Anschließend folgt der eigentliche
Vergleich, dazu werden die jeweiligen Eigenschaften kurz erläutert und danach
wie die jeweiligen Systeme dies umsetzen.

\section{NoSql}
NoSql-Systeme sind keine neue Erfindung, sondern entwickelten sich quasi
parallel mit den relationalen Systemen. Der Grund warum erst jetzt NoSql einen
so großen Durchbruch erfährt, hängt vor allem mit dem BigData-Gedanken zusammen.
Vor BigData war es üblich die Geschäftsdaten auf wenigen Datenbank-Server zu
speichern. Mit der Entwicklung der Sozialen-Netzwerke und der zunehmenden
Vernetzung von Alltagsgegenständen stieg die erzeugte Datenmenge rapide auf
Peta-, Exa-Byte Bereich an. Dies hatte zur Folge, dass sich die Geschäftsdaten
mit herkömmlichen Mitteln nicht mehr effektiv Speichern und verarbeiten ließen
und man nach Alternativen gesucht hat, den NoSql-Systemen.
Obwohl der NoSql Begriff heute so eine Aktualität besitzt, gibt es keine einheitliche
Begriffsdefinition. Vielmehr ist NoSql ähnlich wie \gls{AJAX} ein Sammelbegriff für
verschiedene Technologien und Ideen. Während früher NoSql wörtlich für „no Sql“
also "kein Sql" stand steht es heute eher für „not only Sql“ also "nicht nur Sql".
Auch ab wann man ein System als NoSql-System zählt, ist nicht klar geregelt.
Vielmehr gibt es eine Reihe von Punkten, die ein System erfüllen kann wie: \cite{Edlich2011}

\begin{itemize}
\item Das zugrundeliegende Datenmodell ist nicht relational.
\item Die System sind von Anbeginn auf eine verteilelte und horizontale
Skallierbarkeit ausgerichtet.
\item Das NoSql-System ist Open-Source.
\item Das System ist schemafrei oder hat nur schwache Schemarestriktionen.
\item Aufgrund der verteielten Architektur unterstützt das System einfache
Datenreplikation.
\item Das System besitz eine einfache \gls{API}.
\item Dem System liegt meist auch ein anderes Konsitenzmodell zugrunde:
Eventuelly Consistenz und \gls{BASE} aber nicht \gls{ACID}.
\end{itemize}

. Daraus haben sich die unterschiedlichsten Systeme entwickelt.

\subsection{Entwicklungsmeilensteine}
Der Ursprung von NoSql ist die im Jahr 1979 von Ken Thompson entwickelte
Datenbank DBM\footnote{\url{http://www.gnu.org/software/gdbm/gdbm.html}}, die
bis heute noch von Linux verwendet wird. Die ersten NoSql Datenbanken, die
heute noch existieren entstanden in den 80er-Jahren, wie Lotus Note\footnote{\url{http://www-03.ibm.com/software/products/de/ibmdomino}},
BerkleyDB\footnote{\url{http://www.oracle.com/us/products/database/berkeley-db/index.html}},
GT.M\footnote{\url{https://sourceforge.net/projects/fis-gtm/https://sourceforge.net/projects/fis-gtm/}}, \ldots .
Der Begriff NoSql wurde das erste Mal 1998 von Carlo Strozzi benutzt, der eine
relationale Datenbank entwickelt hatte jedoch ohne Sql. Der richtige Durchbruch
kam allerdings erst im Jahr 2000 mit dem Web 2.0 und den daraus resultierenden
Anstieg des Datenvolumens. Dieser Datenanstieg erforderte neue Methoden zur
effizienteren Speicherung und Verarbeitung. Google war dabei der Vorreiter mit
seinem Map-/Reduce-Ansatz. Die Idee dabei ist es, eine große Datenmenge in
kleinere Pakete aufzuteilen und diese dann unabhängig voneinander zu verarbeiten.
Anschließend werden die einzelnen Zwischenergebnisse gesammelt und zu einem
Gesamtergebnis zusammengefasst. Diese Idee ließ sich sehr gut mit den Ideen
der funktionalen Programmiersprachen umsetzen, denn dort wird auf einer Kopie
der tatsächlichen Daten gearbeitet. Später zogen die anderen großen Firmen mit
eigenen Lösungen nach und heute kann man sagen, dass es den Map-/Reduce-Ansatz
nicht mehr gibt, sondern verschiedene Implementierungen, die für einen gewissen
Einsatzzweck entwickelt wurden. Die Entwicklung der modernen NoSql-Systemen
begann im Jahr 2005 mit Systemen wie Neo4j, Redis, Cassandra, \ldots und hält
bis heute noch an.

\subsection{Arten von NoSql-Systemen}
Wie oben schon beschrieben existieren eine Vielzahl an unterschiedlichsten
NoSql-Systemen. Dies hat zur Folge, dass es nicht leicht ist das für seinen
Anwendungsfall passende System zu finden. Deshalb kam der Gedanke NoSql-Systeme
nach bestimmten Kriterien zu sortieren um eine bessere Entscheidung treffen zu
können. Schnell entwickelten sich mehrere Kriterien. Die gröbste Einteilung,
die man vorgenommen hat, war die Einteilung in Kern-NoSql-Systeme und
Soft-NoSql-Systemen. Wobei die Grenze und der Unterschied zwischen den Gruppen
nicht klar geregelt ist. Danach werden NoSql-Systeme häufig nach ihrem
zugrundeliegenden Datenmodell eingeteilt. Nachfolgend eine Einteilung der
NoSql-Systeme nach dem Datenmodell mit einer Erklärung was man sich unter dem
jeweiligen Typ vorzustellen hat und einige Vertreter davon:

\begin{itemize}
\item Kern-NoSql-Systeme
\begin{description}
\item[dokumentenbasierte Systeme] Dokumentenbasierte System speichern die Daten
in Dateien, meist im \gls{JSON}-Format, ab. Zu dieser Gruppe gehören Systeme wie
MongoDB oder CouchDB.
\item[Schlüssel-Wert-Systeme] Schlüssel-Wert-Systeme speichern wie der Name schon
sagt die Daten unter einem gewissen Schlüssel ab, über den dann später die Daten
wieder gelesen werden. Zu dieser Gruppe gehören Systeme wie Redis, Voldemort,
Memcached.
\item[graphbasierte Systeme] Bei graphbasierten Systemen liegen die Daten als
Graph vor. Solche Systeme ermöglichen es leicht herauszufinden, welcher Knoten
mit einem anderen Knoten in Beziehung steht. Dies wird bei Sozialennetzwerken
bezutzt um herauszufinden, wer "Freund" von jemanden ist. Zu dieser Gruppe
gehören Systeme wie Neo4j.
\item[spaltenorientierte Systeme] Spaltenorientiert System haben Ähnlichkeiten
mit relationalen Systemen, die ja auch über Spalten verfügen. Allerdings kann man
bei einem spaltenorientierten System eine Liste von Schlüssel-Wert-Paaren
abspeichern und im Extremfall kann sogar die Spalte eine Liste von
Schlüssel-Wert-Paaren sein. Zu dieser Gruppe gehören Systeme wie HBase, Cassandra.
\end{description}
\item Soft-NoSql-Systeme
\begin{description}
\item[XML-Datenbanken] XML-Datenbanken sind spezielle dokumentenbasierte Systeme
bei denen statt \gls{JSON}-Dateien XML-Dateien benutzt werden.
\item[Objekt-Datenbanken] Objekt-Datenbanken speichern Objektstrukturen ab.
Vorraussetzung ist allerdings, das die Objekte serialisierbar sind. Werden
solche Objekt-Datenbanken anstelle von relationalen Datenbanken eingesetzt, dann
kann man sich den Einsatz von \gls{ORM}-Frameworks wie Hibernate oder EclipseLink
sparen.
\end{description}
\end{itemize}

. Im weiteren Verlauf werden wir nur noch Schlüssel-Wert-Systeme betrachten.
